\documentclass[12pt]{tdtp}
\usepackage{hyperref}
\usepackage{tabularx,colortbl}
\usepackage{multirow}
\usepackage{listings}
\lstset{
	language=VHDL,
basicstyle=\tiny\ttfamily}
\definecolor{light-gray}{gray}{0.96}
\definecolor{pageheading-gray}{gray}{0.2}
\definecolor{dark-gray}{gray}{0.45}
\definecolor{dark-green}{rgb}{0.245,0.121,0.0}

\newcommand{\auteur}{Cedric Lemaitre}
\newcommand{\couriel}{c.lemaitre58@gmail.com}
\newcommand{\promo}{BScv}
\newcommand{\annee}{2017-2018}
\newcommand{\matiere}{Computer science}

\newcommand{\tdtp}{Lab \#3}
\renewcommand{\sujet}{Function, array, matrix}


\begin{document}
\titre
\textit{NB : use good practice for naming and write code \footnote{\url{https://google.github.io/styleguide/cppguide.html}}!!!}

%%%%%%%%%%%
\Exo

Download data from this place : \url{https://raw.githubusercontent.com/mwaskom/seaborn-data/master/titanic.csv}
as *.csv file.

Create an object which allow to stock the allowing attributes :
\begin{itemize}
	\item survived
	\item pclass
	\item sex
	\item age
	\item sibsp
	\item parch
	\item fare
	\item embarked
	\item class
	\item who
	\item adult\_male
	\item deck
	\item embark\_town
	\item alive
	\item alone
\end{itemize}

Choose the great type for each attribute considering data in csv file.

In your main file, create a std::vector of your object, which object should contain a line of the file.
Use string, ifstream object in order to manipulate data in the file.
Be carefull when data is missing, you have to add default or special values.


\Exo

Create an object which allows to process the data store in your previous vector of object.
Add to this class, the following functions :

\begin{itemize}
	\item mean age value
	\item number alive
	\item histogram of embarked town
\end{itemize}

Test all this function in your main function.

\end{document}
